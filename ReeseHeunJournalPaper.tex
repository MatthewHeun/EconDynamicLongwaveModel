\documentclass[letterpaper,12pt]{article}
\usepackage{dgjournal}          
\usepackage{mathptmx}
\usepackage{graphics}
\usepackage[authoryear,comma,longnamesfirst,sectionbib]{natbib} 

\begin{document}

%% Do NOT include any fronmatter information; including the title, author names,
%% institutes, acknowledgments and title footnotes (author information, funding
%% sources, etc.). Start the document with the first section or paragraph of
%% the article.

\section{Mathematical model from Jones (2001)}

\begin{equation}
x + y = 6
\end{equation}

\subsection{A second order heading}

Some text under the subheading. Paragraphs that follow heads are not
indented.

Math should also be set in Times. Use the mathptmx package if you do not have
any of the commercially available fonts that are compatible with Times.
\begin{equation}
    y^{(n)} = \sum_{i=0}^{n-1} a_i(x) y^{(i)} + r(x) 
\end{equation}

All environments provided by the standard LaTeX document classes are
unchanged. Vertical spaces within lists have been altered to comply with De Gruyter
requirements.
\begin{enumerate}
\item This is the first item within the list. Some more text here in order to
  display the alignment.
\item Another item in the list.
\item Yet another item in the list.
\end{enumerate}

Here is an example of a Figure. It's the same as in standard LaTeX.

\begin{figure}[!h]
%% Use the graphics package to insert figures
%% \includegraphics{figure.eps}
% Use \centering to center the table
\centering
%% A small box in place of a figure
\framebox{%
  \begin{minipage}{10pc}
    \begin{center}
      \vspace{1cm}\par
      A figure\par
      \vspace{1cm}
    \end{center}
  \end{minipage}}
\caption{Insert your caption here. If you wish to label your figure for
  cross-referencing, use a label either within the caption or after it.}
\label{fig1}
\end{figure}

An example of a table follows. This is also the same as in standard LaTeX.

\begin{table}[!h]
% Use \centering to center the table
\centering
\caption{Insert your table caption here. If you wish to label the table for
  cross-referencing, use a label either within the caption or after it.}
\begin{tabular}{llll}
\hline
Symbol        & LaTeX Command      & Symbol      & LaTeX Command \\
\hline
$\alpha$      & \verb+\alpha+      & $\zeta$     & \verb+\zeta+ \\
$\beta$       & \verb+\beta+       & $\eta$      & \verb+\eta+ \\
$\gamma$      & \verb+\gamma+      & $\theta$    & \verb+\theta+ \\
$\delta$      & \verb+\delta+      & $\vartheta$ & \verb+\vartheta+ \\
$\epsilon$    & \verb+\epsilon+    & $\iota$     & \verb+\iota+ \\
$\varepsilon$ & \verb+\varepsilon+ & $\kappa$    & \verb+\kappa+ \\
\hline
\end{tabular}
\end{table}

Use the \verb+thebibliography+ environment for the references.  BibTeX users may
use the provided BibTeX style file DeGruyter.bst.

%% BibTeX support
\bibliographystyle{DeGruyter}
\bibliography{sample}

\end{document}