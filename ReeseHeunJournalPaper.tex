\documentclass[letterpaper,12pt]{article}
\usepackage{dgjournal}          
\usepackage{mathptmx}
\usepackage{graphics}
\usepackage[authoryear,comma,longnamesfirst,sectionbib]{natbib} 

\begin{document}

%% Do NOT include any fronmatter information; including the title, author names,
%% institutes, acknowledgments and title footnotes (author information, funding
%% sources, etc.). Start the document with the first section or paragraph of
%% the article.

\section{Mathematical model from Jones (2001) [take 2]}
The longwave growth model used as a basis for this project was taken from Jones (2001). The production function that Jones is using to ultimately predict output is given by equation \ref{eq:Jones_production_function} and is a function of the stock of ideas ($a_\mathrm{t}$), labor in the goods producing sector ($l_\mathrm{Y,t}$), land ($t_\mathrm{t}$), and an exogenous shock parameter ($\epsilon_\mathrm{t}$).

\begin{equation} \label{eq:Jones_production_function}
y_\mathrm{t} = a_\mathrm{t} ^\sigma l_\mathrm{Y,t} ^\beta t_\mathrm{t} ^{1-\beta} \epsilon_\mathrm{t}
\end{equation}

The production function variables are indexed, referenced against some base year to non-dimensionalize them, and given by the following definitions. Where $Y$ represents the output (\$/year), $A$ represents the stock of knowledge and ideas (ideas/year), and $t$ represents the stock of available land (land/year).

\begin{equation} \label{eq:index_y}
y \equiv \frac{Y}{Y_\mathrm{0}},
\end{equation}
\begin{equation} \label{eq:index_a}
a \equiv \frac{A}{A_\mathrm{0}},
\end{equation}
\begin{equation} \label{eq:index_t}
t \equiv \frac{T}{T_\mathrm{0}},
\end{equation}

%%BRIDGE SECTIONS BETWEEN PRODUCTION FUNCTION AND UTILITY FUNCTION

The utility function from Jones (2001) was the basis for our model and is of the form found in equation \ref{eq:utility_function}. The utility function is used to represent an individuals preferences towards consumption or having children at a point in time. 

\begin{equation} \label{eq:utility_function}
u(c_\mathrm{t}, b_\mathrm{t}) = \frac{1-\mu}{1-\gamma} \bigg{(}\frac{k_\mathrm{c} \tilde c_\mathrm{t}}{U_\mathrm{c,0}} \bigg{)}^{1-\gamma} + \frac{\mu}{1-\eta} \bigg{(}\frac{k_\mathrm{b} \tilde b_\mathrm{t}}{U_\mathrm{b,0}} \bigg{)}^{1-\eta},
\end{equation}

Where $\mu$, $\gamma$, and $\eta$ are dimensionless parameters having values between 0 and 1. $k_\mathrm{c}$ and $k_\mathrm{b}$ represent the amount of utility an individual obtains per unit consumption or birth, in units of utils/\$ and utils/birth respectively. The variable $\tilde c_\mathrm{t}$ is the amount of the individual's consumption in \$/year-person and $\tilde b_\mathrm{t}$ is the number of births in births/year-person. Both the consumption and birth utility terms are divided by the utility in the first year of the simulation, $U_\mathrm{0}$, which is taken to be the year 25,000 B.C. from Jones. It is also important to note that $\tilde c_\mathrm{t}$ is the amount of consumption above a certain subsistence level and $\tilde b_\mathrm{t}$ represents the amount of births above some long-run birth rate, defined by equation \ref{eq:c_tilde} and equation \ref{eq:b_tilde}.

\begin{equation} \label{eq:c_tilde}
\tilde c_\mathrm{t} \equiv c_\mathrm{t} - \bar c,
\end{equation}

\begin{equation} \label{eq:b_tilde}
\tilde b_\mathrm{t} \equiv b_\mathrm{t} - \bar b,
\end{equation}

Where $\bar c$ is the subsistence level of consumption and $\bar b$ is the long term birth rate.

The utility function is subject to the following first order condition.

\begin{equation} \label{eq:first_order_condition}
\frac{\partial u/ \partial\tilde b}{\partial u/ \partial\tilde c} = \frac{w_\mathrm{t}}{\alpha},
\end{equation}

Where $w_\mathrm{t}$ is the wage rate that a person is paid in \$/year and $\alpha$ is an equivalent "wage rate" on the number of births per year, or an equivalent monetary amount that a person derives from each birth in \$/year. The appearance of these two coefficients are from the time constraints in the economy. These time constraints are given by equations \ref{eq:pop_work}, \ref{eq:consumption_constraint}, and \ref{eq:birth_constraint}.

\begin{equation}\label{eq:pop_work}
L = \tau_\mathrm{t} N,
\end{equation}

\begin{equation} \label{eq:consumption_constraint}
c_\mathrm{t} = w_\mathrm{t} \tau_\mathrm{t},
\end{equation}

\begin{equation} \label{eq:birth_constraint}
b_\mathrm{t} = \alpha (1-\tau_\mathrm{t}),
\end{equation}

Where $\tau$ is a number between 0 and 1 representing the fraction of available time that people spend on labor (unitless), $N$ is the total population, and $L$ is the total amount of available labor in the economy. So, $(1-\tau)$ represents the fraction of available time that's spent on birth.

The time constraints, in conjunction with the first order condition, reduce the utility function down to a form that directly relates birth and consumption, as shown in equation \ref{eq:FOC_and_time_constraints}.

\begin{equation} \label{eq:FOC_and_time_constraints}
\frac{k_\mathrm{b}}{U_\mathrm{b,0}} \tilde b_\mathrm{t} = \bigg{[} \frac{k_\mathrm{b}}{U_\mathrm{b,0}} \frac{U_\mathrm{c,0}}{k_\mathrm{c}} \frac{\alpha \mu}{1-\mu} \frac{( \frac{k_\mathrm{c}}{U_\mathrm{c,0}} \tilde c_\mathrm{t} )^\gamma}{w_\mathrm{t}}\bigg{]}^{1/\eta} ,
\end{equation}

Where both sides of this equation are dimensionless (to allow application of the exponential factor).

The labor section of the economy can be split into two different sectors; one devoted to innovation and producing new ideas and one devoted to producing goods.


\section{Mathematical model from Jones (2001)}

The mathematical model being used for this project was taken from Jones (2001). The production function that Jones uses is defined in 
Definitions of the production function variables, indexed against some base year.

Equations representing compensation for innovation in the knowledge and labor sector respectively.
\begin{equation}
w_\mathrm{At} L_\mathrm{At} = \pi_\mathrm{t} Y_\mathrm{t}
\end{equation}

\begin{equation}
w_\mathrm{Yt} L_\mathrm{Yt} = (1-\pi_\mathrm{t})Y_\mathrm{t}
\end{equation}

Also, in static equilibrium we expect:
\begin{equation}
w_\mathrm{A} = w_\mathrm{Y} = w_\mathrm{t}
\end{equation}

\begin{equation}
\pi = \frac{L_\mathrm{At}}{L}
\end{equation}

Differential equation representing the change in the stock of knowledge over time.
\begin{equation} \label{eq:da_dt}
\frac{da}{dt} = \delta l_\mathrm{At}^\lambda a_\mathrm{t}^\phi
\end{equation}

The utility function is subject to the first-order condition.


The first order condition, in conjunction with the time constraints, implies that,

Indexed working population.
\begin{equation}
n \equiv \frac{N}{N_\mathrm{0}}
\end{equation}

Differential equation representing the change in the population over time.
\begin{equation}
\frac{dn}{dt} = b_\mathrm{t} n_\mathrm{t} - d_\mathrm{t} n_\mathrm{t}
\end{equation}

Death rate equation.
\begin{equation} \label{deathrate}
d_\mathrm{t} = \frac{1}{\omega_\mathrm{1} z_\mathrm{t}^{\omega_\mathrm{2}} + \omega_\mathrm{3} z_\mathrm{t}} + \bar d
\end{equation}

\subsection{A second order heading}

Some text under the subheading. Paragraphs that follow heads are not
indented.

Math should also be set in Times. Use the mathptmx package if you do not have
any of the commercially available fonts that are compatible with Times.
\begin{equation}
    y^{(n)} = \sum_{i=0}^{n-1} a_i(x) y^{(i)} + r(x) 
\end{equation}

All environments provided by the standard LaTeX document classes are
unchanged. Vertical spaces within lists have been altered to comply with De Gruyter
requirements.
\begin{enumerate}
\item This is the first item within the list. Some more text here in order to
  display the alignment.
\item Another item in the list.
\item Yet another item in the list.
\end{enumerate}

Here is an example of a Figure. It's the same as in standard LaTeX.

\begin{figure}[!h]
%% Use the graphics package to insert figures
%% \includegraphics{figure.eps}
% Use \centering to center the table
\centering
%% A small box in place of a figure
\framebox{%
  \begin{minipage}{10pc}
    \begin{center}
      \vspace{1cm}\par
      A figure\par
      \vspace{1cm}
    \end{center}
  \end{minipage}}
\caption{Insert your caption here. If you wish to label your figure for
  cross-referencing, use a label either within the caption or after it.}
\label{fig1}
\end{figure}

An example of a table follows. This is also the same as in standard LaTeX.

\begin{table}[!h]
% Use \centering to center the table
\centering
\caption{Insert your table caption here. If you wish to label the table for
  cross-referencing, use a label either within the caption or after it.}
\begin{tabular}{llll}
\hline
Symbol        & LaTeX Command      & Symbol      & LaTeX Command \\
\hline
$\alpha$      & \verb+\alpha+      & $\zeta$     & \verb+\zeta+ \\
$\beta$       & \verb+\beta+       & $\eta$      & \verb+\eta+ \\
$\gamma$      & \verb+\gamma+      & $\theta$    & \verb+\theta+ \\
$\delta$      & \verb+\delta+      & $\vartheta$ & \verb+\vartheta+ \\
$\epsilon$    & \verb+\epsilon+    & $\iota$     & \verb+\iota+ \\
$\varepsilon$ & \verb+\varepsilon+ & $\kappa$    & \verb+\kappa+ \\
\hline
\end{tabular}
\end{table}

Use the \verb+thebibliography+ environment for the references.  BibTeX users may
use the provided BibTeX style file DeGruyter.bst.

%% BibTeX support
\bibliographystyle{DeGruyter}
\bibliography{sample}

\end{document}