\documentclass[letterpaper,12pt]{article}
\usepackage{dgjournal}          
\usepackage{mathptmx}
\usepackage{graphics}
\usepackage[authoryear,comma,longnamesfirst,sectionbib]{natbib} 

\begin{document}

%% Do NOT include any frontmatter information; including the title, author names,
%% institutes, acknowledgments and title footnotes (author information, funding
%% sources, etc.). Start the document with the first section or paragraph of
%% the article.

\section{Longwave growth model}
In our model, adapted from the longwave growth model of \citet{Jones:2001wn}, economic production is given by

\begin{equation} \label{eq:Jones_production_function}
	y_\mathrm{t} = a_\mathrm{t} ^\sigma l_\mathrm{Y,t} ^\beta t_\mathrm{t} ^{1-\beta} \epsilon_\mathrm{t}
\end{equation}

\noindent where $a_\mathrm{t}$ is the indexed stock of ideas, $l_\mathrm{Y,t}$ is indexed labor in the goods producing sector, $t_\mathrm{t}$ is indexed land (always 1.0), and $\epsilon_\mathrm{t}$ is an exogenous shock parameter. 

Lowercase variables represent unitless indexed variables. The indexed factors of production ($a$, $l$, and $t$) at time $\mathrm{t}$ are non-dimensionalized by indexing relative to a base year (subscript ``0''):

\begin{equation} \label{eq:index_y}
	y_\mathrm{t} \equiv \frac{Y_\mathrm{t}}{Y_\mathrm{0}}
\end{equation}

\begin{equation} \label{eq:index_a}
	a_\mathrm{t} \equiv \frac{A_\mathrm{t}}{A_\mathrm{0}}
\end{equation}

\begin{equation} \label{eq:index_l}
	l_\mathrm{Y,t} \equiv \frac{L_\mathrm{Y,t}}{L_\mathrm{Y,0}}
\end{equation}

\begin{equation} \label{eq:index_t}
	t_\mathrm{t} \equiv \frac{T_\mathrm{t}}{T_\mathrm{0}}
\end{equation}

\noindent with $Y$ representing economic output (in \$/year), $A$ representing the stock of knowledge and ideas (in ideas/year), $L_\mathrm{Y}$ representing labor in the production sector (in workers/year), and $T$ representing the stock of available land ($\mathrm{m}^2$). [*********** Is $A$ really in ideas/year? Or is it really in ideas? --MKH ************]

Our model for preferences between consumption and childrearing is based on the utility function from \citet{Jones:2001wn}: ***************** Caleb: I'm adding $u_c$ and $u_b$ here. Do you agree? We may also need to define $u_c$ and $u_b$ with equations.  --MKH ************

\begin{equation} \label{eq:utility_function}
	u(c_\mathrm{t}, b_\mathrm{t}) = u_\mathrm{c} + u_\mathrm{b} = \frac{1-\mu}{1-\gamma} \left(\frac{k_\mathrm{c} \tilde c_\mathrm{t}}{U_\mathrm{c,0}} \right)^{1-\gamma} + \frac{\mu}{1-\eta} \left(\frac{k_\mathrm{b} \tilde b_\mathrm{t}}{U_\mathrm{b,0}} \right)^{1-\eta}
\end{equation}

\noindent where $\mu$, $\gamma$, and $\eta$ are dimensionless parameters having values between 0 and 1; $k_\mathrm{c}$ and $k_\mathrm{b}$ represent the utility an individual obtains per unit consumption or birth (in units of utils/\$ and utils/birth, respectively); and $U_{\mathrm{c,0}}$ and $U_{\mathrm{b,0}}$ are the utility derived from consumption and childrearing in the first year of the simulation, respectively (in units of utils/year). ************* Caleb: Need to define $U_{b,0}$ and $U_{c,0}$ with equations here --MKH ****************** The variable $\tilde c_\mathrm{t}$ is the amount of the individual's consumption in \$/year-person and $\tilde b_\mathrm{t}$ is the number of births in births/year-person. $\tilde c_\mathrm{t}$ represents the amount of consumption above subsistence level ($\bar c$ in \$/year-person) and $\tilde b_\mathrm{t}$ represents the amount of births above some long-run birth rate ($\bar b$, in births/year per capita).

\begin{equation} \label{eq:c_tilde}
	\tilde c_\mathrm{t} \equiv c_\mathrm{t} - \bar c
\end{equation}

\begin{equation} \label{eq:b_tilde}
	\tilde b_\mathrm{t} \equiv b_\mathrm{t} - \bar b
\end{equation}

The first order condition on the utility function indicates that a person will obtain equal incremental utility from spending additional time consuming or rearing children: ******************* Caleb: Is this next equation correct? --MKH *********************

\begin{equation} \label{eq:first_order_condition_def}
	\frac{\partial u_{c}}{\partial L} = \frac{\partial u_{b}}{\partial L}
\end{equation}

\noindent The first order condition simplifies to 

\begin{equation} \label{eq:first_order_condition_simplified}
	\frac{\partial u/ \partial\tilde b}{\partial u/ \partial\tilde c} = \frac{w_\mathrm{t}}{\alpha},
\end{equation}

\noindent where $w_\mathrm{t}$ is the wage rate that a worker is paid (in \$/year-person) and $\alpha$ is a birth efficiency (in births/year per capita). [********** Caleb: notice that I'm using ``per capita'' in several places. Is that appropriate? --MKH ********************] $w_\mathrm{t}$ and $\alpha$ come from the time constraints in the economy. These time constraints are given by:

\begin{equation}\label{eq:pop_work}
	L = \tau_\mathrm{t} N
\end{equation}

\begin{equation} \label{eq:consumption_constraint}
	c_\mathrm{t} = w_\mathrm{t} \tau_\mathrm{t}
\end{equation}

\begin{equation} \label{eq:birth_constraint}
	b_\mathrm{t} = \alpha (1-\tau_\mathrm{t})
\end{equation}

\noindent where $\tau$ is a number between 0 and 1 representing the fraction of available time that people spend on labor (unitless), $N$ is the total population (in persons), and $L$ is the total amount of available labor in the economy (in persons). So, $(1-\tau)$ represents the fraction of available time spent on childrearing.

Applying the first order condition (Equation \ref{eq:first_order_condition_simplified}) and including time constraints yields an equation that describes preferences for consumption and childrearing:

\begin{equation} \label{eq:FOC_and_time_constraints}
	\frac{1}{\alpha \mu} \frac{U_\mathrm{b,0}}{k_\mathrm{b}} \left( \frac{k_\mathrm{b}}{U_\mathrm{b,0}} \tilde b_\mathrm{t} \right) ^{\eta} 
	= \frac{1}{w_\mathrm{t}(1-\mu)} \frac{U_\mathrm{c,0}}{k_\mathrm{c}}  \left( \frac{k_\mathrm{c}}{U_\mathrm{c,0}} \tilde c_\mathrm{t} \right)^\gamma .
\end{equation}

Labor is split between two sectors; one devoted to innovation and producing new ideas ($A$) and one devoted to producing goods ($Y$). The split between these sectors is based on the wages paid to each as shown in the equations below:

\begin{equation} \label{eq:knowledge_comp}
	w_\mathrm{A,t} L_\mathrm{A,t} = \pi_\mathrm{t} Y_\mathrm{t}
\end{equation}

\begin{equation} \label{eq:labor_comp}
	w_\mathrm{Y,t} L_\mathrm{Y,t} = (1-\pi) Y_\mathrm{t}
\end{equation}

\noindent where $w_\mathrm{A,t}$ and $w_\mathrm{Y,t}$ are the wages paid to workers in the knowledge and goods producing sectors respectively (in \$/person-year), and $L_\mathrm{A,t}$ and $L_\mathrm{Y,t}$ represent the number of workers employed by each of the sectors (in persons/year). The total labor supply is constrained as the sum of the two sectors.

\begin{equation} \label{labor_supply}
	L_\mathrm{t} = L_\mathrm{A,t} + L_\mathrm{Y,t}
\end{equation}

The two labor supplies can also indexed when necessary.

\begin{equation}
	l_\mathrm{Y,t} \equiv \frac{L_\mathrm{Y,t}}{L_\mathrm{Y,0}}
\end{equation}

\begin{equation}
	l_\mathrm{A,t} \equiv \frac{L_\mathrm{A,t}}{L_\mathrm{A,0}}
\end{equation}

In equilibrium, it is expected that the wages will be equal.

\begin{equation} \label{eq:wage_equality}
	w_\mathrm{A,t} = w_\mathrm{Y,t} = w_\mathrm{t}
\end{equation}

The dimensionless variable $\pi_\mathrm{t}$ is the fraction of the economy's total output ($Y_\mathrm{t}$) that is devoted to compensating inventors. The wage equality allows $\pi_\mathrm{t}$ to be directly proportional to the fraction of the total employed population working in the knowledge sector in static equilibrium.

\begin{equation} \label{eq:pi}
	\pi_\mathrm{t} = \frac{L_\mathrm{A,t}}{L_\mathrm{t}}
\end{equation}

Two differential equations describe the growth of knowledge and population over time in the production function (Equation \ref{eq:Jones_production_function}), because population directly affects the labor force. The rate of change in the indexed stock of knowledge ($a_{\mathrm{t}}$) is given by

\begin{equation} \label{eq:da_dt}
	\frac{da}{dt} = \delta l_\mathrm{A,t}^\lambda a_\mathrm{t}^\phi,
\end{equation}

\noindent where $\delta$, $\lambda$, and $\phi$ are constants, $l_\mathrm{A,t}$ is the indexed labor in the knowledge sector of the economy, and $a_\mathrm{t}$ is the indexed stock of ideas.

The rate of change of indexed population ($n$) is a function of birth and mortality rates ($b$ and $d$ in births/year per capita and deaths/year per capita, respectively):

\begin{equation} \label{eq:dn_dt}
	\frac{dn}{dt} = b_\mathrm{t} n_\mathrm{t} - d_\mathrm{t} n_\mathrm{t},
\end{equation}

\noindent where $b_\mathrm{t}$ is the birth rate from Equation \ref{eq:utility_function}, $n_\mathrm{t}$ is the indexed population, and $d_\mathrm{t}$ is the mortality rate in deaths/person-year. The mortality rate is determined from a fit of historical mortality rate data \citep{Jones:2001wn} and is given by

\begin{equation} \label{eq:mortality_rate}
	d_\mathrm{t} = \frac{1}{\omega_\mathrm{1} z_\mathrm{t}^{\omega_\mathrm{2}} + \omega_\mathrm{3} z_\mathrm{t}} + \bar d,
\end{equation}

\noindent where $d_\mathrm{t}$ is the mortality rate and $\bar d$ is the long-run mortality rate, both in deaths/year-person. $\omega_\mathrm{1}$, $\omega_\mathrm{2}$, and $\omega_\mathrm{3}$ are fitted parameters. The ratio of consumption to the subsistence level is given by $z_\mathrm{t}$:

\begin{equation} \label{eq:z}
	z_\mathrm{t} = \frac{c_\mathrm{t}}{\bar c} - 1, 
\end{equation}

\noindent which is an indicator of the mortality rate through Equation \ref{eq:mortality_rate}.


\subsection{A second order heading}

Some text under the subheading. Paragraphs that follow heads are not
indented.

Math should also be set in Times. Use the mathptmx package if you do not have
any of the commercially available fonts that are compatible with Times.
\begin{equation}
    y^{(n)} = \sum_{i=0}^{n-1} a_i(x) y^{(i)} + r(x) 
\end{equation}

All environments provided by the standard LaTeX document classes are
unchanged. Vertical spaces within lists have been altered to comply with De Gruyter
requirements.
\begin{enumerate}
\item This is the first item within the list. Some more text here in order to display the alignment.
\item Another item in the list.
\item Yet another item in the list.
\end{enumerate}

Here is an example of a Figure. It's the same as in standard LaTeX.

\begin{figure}[!h]
%% Use the graphics package to insert figures
%% \includegraphics{figure.eps}
% Use \centering to center the table
\centering
%% A small box in place of a figure
\framebox{%
  \begin{minipage}{10pc}
    \begin{center}
      \vspace{1cm}\par
      A figure\par
      \vspace{1cm}
    \end{center}
  \end{minipage}}
\caption{Insert your caption here. If you wish to label your figure for
  cross-referencing, use a label either within the caption or after it.}
\label{fig1}
\end{figure}

An example of a table follows. This is also the same as in standard LaTeX.

\begin{table}[!h]
% Use \centering to center the table
\centering
\caption{Insert your table caption here. If you wish to label the table for
  cross-referencing, use a label either within the caption or after it.}
\begin{tabular}{llll}
\hline
Symbol        & LaTeX Command      & Symbol      & LaTeX Command \\
\hline
$\alpha$      & \verb+\alpha+      & $\zeta$     & \verb+\zeta+ \\
$\beta$       & \verb+\beta+       & $\eta$      & \verb+\eta+ \\
$\gamma$      & \verb+\gamma+      & $\theta$    & \verb+\theta+ \\
$\delta$      & \verb+\delta+      & $\vartheta$ & \verb+\vartheta+ \\
$\epsilon$    & \verb+\epsilon+    & $\iota$     & \verb+\iota+ \\
$\varepsilon$ & \verb+\varepsilon+ & $\kappa$    & \verb+\kappa+ \\
\hline
\end{tabular}
\end{table}

Use the \verb+thebibliography+ environment for the references.  BibTeX users may
use the provided BibTeX style file DeGruyter.bst.

%% BibTeX support
\bibliographystyle{DeGruyter}
\bibliography{ReeseHeunJournalPaper}

\end{document}