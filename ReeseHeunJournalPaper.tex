\documentclass[letterpaper,12pt]{article}
\usepackage{dgjournal}          
\usepackage{mathptmx}
\usepackage{graphicx} % Allows use of width and scale options for \includegraphics
\usepackage[authoryear,comma,longnamesfirst,sectionbib]{natbib} 
%% The lineno packages adds line numbers. Start line numbering with
%% \begin{linenumbers}, end it with \end{linenumbers}. Or switch it on
%% for the whole article with \linenumbers after \end{frontmatter}.
\usepackage{lineno}
\linenumbers % Turns on line numbering for the entire document.

\begin{document}

%% Do NOT include any frontmatter information; including the title, author names,
%% institutes, acknowledgments and title footnotes (author information, funding
%% sources, etc.). Start the document with the first section or paragraph of
%% the article.

%%%%%%%%%%%%%%%%%%%%%%%%%%%%%%%
\section*{Abstract}
%%%%%%%%%%%%%%%%%%%%%%%%%%%%%%%


%%%%%%%%%%%%%%%%%%%%%%%%%%%%%%%
\section{Introduction}
\label{sec:Introduction}
%%%%%%%%%%%%%%%%%%%%%%%%%%%%%%%


%%%%%%%%%%%%%%%%%%%%%%%%%%%%%%%
\section{Sources of data}
\label{sec:Sources_of_data}
%%%%%%%%%%%%%%%%%%%%%%%%%%%%%%%

**** Caleb complete this section ****


%%%%%%%%%%%%%%%%%%%%%%%%%%%%%%%
\section{Basic model (without energy and capital stock)}
\label{sec:Basic_model}
%%%%%%%%%%%%%%%%%%%%%%%%%%%%%%%

The model is adapted from the longwave growth model of \citet{Jones:2001wn}. 

%++++++++++++++++++++++++++++++
\subsection{Basic model development}
\label{sec:Basic_model_development}
%++++++++++++++++++++++++++++++

**** Caleb update this section ****

The basic model includes neither energy flows nor capital stock.

\begin{figure} \label{fig:ModelWithoutEnergy}
  \begin{center}
    \includegraphics[width=\textwidth]{figure_other/ModelWithoutEnergy.pdf}
    \caption{Dynamic model without energy.}
  \end{center}
\end{figure}

For \citet{Jones:2001wn}, economic production ($y$) is given by

\begin{equation} \label{eq:Jones_production_function}
	y_\mathrm{t} = a_\mathrm{t} ^\sigma l_\mathrm{Y,t} ^\beta t_\mathrm{t} ^{1-\beta} \epsilon_\mathrm{t}
\end{equation}

\noindent where $a$ is the indexed stock of ideas, $l_\mathrm{Y}$ is indexed labor in the goods producing sector, $t$ is indexed land (always 1.0), and $\varepsilon$ is an exogenous shock parameter. The subscript ``t'' indicates time (measured in years). Any symbol in a equation with subscript ``$\mathrm{t}$'' indicates a time-varying quantity. Lowercase variable names represent unitless, indexed quantities. The indexed factors of production ($a$, $l$, and $t$) are defined relative to a base year ($\mathrm{t} = 0$):

\begin{equation} \label{eq:index_y}
	y_\mathrm{t} \equiv \frac{Y_\mathrm{t}}{Y_\mathrm{0}}
\end{equation}

\begin{equation} \label{eq:index_a}
	a_\mathrm{t} \equiv \frac{A_\mathrm{t}}{A_\mathrm{0}}
\end{equation}

\begin{equation} \label{eq:index_l}
	l_\mathrm{Y,t} \equiv \frac{L_\mathrm{Y,t}}{L_\mathrm{Y,0}}
\end{equation}

\begin{equation} \label{eq:index_t}
	t_\mathrm{t} \equiv \frac{T_\mathrm{t}}{T_\mathrm{0}} = 1.0
\end{equation}

\noindent with $Y$ representing economic output (in 2005\$/year), $A$ representing the stock of knowledge and ideas (in ideas), $L_\mathrm{Y}$ representing labor in the production sector (in workers/year), and $T$ representing the stock of available land (in m$^2$).

Our model for preferences between consumption and childrearing is based on the utility function from \citet{Jones:2001wn}:

\begin{equation} \label{eq:utility_function}
	u(c_\mathrm{t}, b_\mathrm{t}) = u_\mathrm{c} + u_\mathrm{b} = \frac{1-\mu}{1-\gamma} \left(\frac{k_\mathrm{c} \tilde c_\mathrm{t}}{U_\mathrm{c,0}} \right)^{1-\gamma} + \frac{\mu}{1-\eta} \left(\frac{k_\mathrm{b} \tilde b_\mathrm{t}}{U_\mathrm{b,0}} \right)^{1-\eta}
\end{equation}

\noindent where $\mu$, $\gamma$, and $\eta$ are dimensionless parameters with values between 0 and 1; $k_\mathrm{c}$ and $k_\mathrm{b}$ represent the utility an individual obtains per unit consumption or birth (in units of utils/\$ and utils/birth, respectively); and $U_{\mathrm{c,0}}$ and $U_{\mathrm{b,0}}$ are the utility derived from consumption and childrearing, respectively, at $\mathrm{t} = 0$ (in units of utils/year). The variable $\tilde c$ is the rate of the individual's consumption in 2005\$/year-capita and $\tilde b$ is the rate of births in births/year-capita. $u_\mathrm{c}$ and $u_\mathrm{b}$ are the utilities obtained from consumption and births, represented by the first and second terms, respectively, on the right side of Equation~\ref{eq:utility_function}. $\tilde c$ represents the rate of consumption above subsistence level ($\bar c$ in 2005\$/year-capita), and $\tilde b$ represents the rate of births above some long-run birth rate ($\bar b$, in births/year-capita).

\begin{equation} \label{eq:c_tilde}
	\tilde c_\mathrm{t} \equiv c_\mathrm{t} - \bar c
\end{equation}

\begin{equation} \label{eq:b_tilde}
	\tilde b_\mathrm{t} \equiv b_\mathrm{t} - \bar b
\end{equation}

The first order condition on the utility function (Equation \ref{eq:utility_function}) requires that a person will obtain equal incremental utility from spending additional time consuming or rearing children:

\begin{equation} \label{eq:first_order_condition_def}
	\frac{\partial u_{c}}{\partial \tau} = \frac{\partial u_{b}}{\partial \tau}
\end{equation}

\noindent where $\tau \in (0,1)$. $\tau$ can be interpreted on a per-person basis as worked hours per total available working time. Or, if each individual in the population is assumed to be either a worker or a childrearer, $\tau$ represents the ratio of workers to total population ($L/N$). The first order condition simplifies to 

\begin{equation} \label{eq:first_order_condition_simplified}
	\frac{\partial u/ \partial\tilde b}{\partial u/ \partial\tilde c} = \frac{w_\mathrm{t}}{\nu},
\end{equation}

\noindent where $w$ is the wage rate (in 2005\$/year-worker) and $\nu$ is a birth efficiency (in births/year-capita). $w$ and $\nu$ come from time constraints **** Are these really both \emph{time} constraint equations? **** in the economy, given by:

\begin{equation}\label{eq:pop_work}
	L_\mathrm{t} = \tau_\mathrm{t} N_\mathrm{t}
\end{equation}

\begin{equation} \label{eq:consumption_constraint}
	c_\mathrm{t} = w_\mathrm{t} \tau_\mathrm{t}
\end{equation}

\begin{equation} \label{eq:birth_constraint}
	b_\mathrm{t} = \nu (1-\tau_\mathrm{t})
\end{equation}

\noindent $N$ is the total population (in persons), and $L$ is the total amount of available labor in the economy (in workers). So, $(1-\tau)$ represents the fraction of available time spent on childrearing (unitless or time childrearing per total time). It is worth noting that the units of $\tau$ can also be defined in terms of workers/capita, as shown in Equation~\ref{eq:pop_work}.

Applying the first order condition (Equation \ref{eq:first_order_condition_simplified}) and including time constraints **** time constraints only? **** yields an equation that describes preferences for consumption and childrearing:

\begin{equation} \label{eq:FOC_and_time_constraints}
	\frac{1}{\alpha \mu} \frac{U_\mathrm{b,0}}{k_\mathrm{b}} \left( \frac{k_\mathrm{b}}{U_\mathrm{b,0}} \tilde b_\mathrm{t} \right) ^{\eta} 
	= \frac{1}{w_\mathrm{t}(1-\mu)} \frac{U_\mathrm{c,0}}{k_\mathrm{c}}  \left( \frac{k_\mathrm{c}}{U_\mathrm{c,0}} \tilde c_\mathrm{t} \right)^\gamma .
\end{equation}

**** Weren't we going to include thg $G$ term in Equation \ref{eq:FOC_and_time_constraints}? Does $\alpha$ belong in the above equation? **** Labor is split between two sectors; one devoted to innovation and producing new ideas ($A$) and the other devoted to producing goods ($Y$). The split between these sectors is based on the wages paid to each as shown in the equations below:

\begin{equation} \label{eq:knowledge_comp}
	w_\mathrm{A,t} L_\mathrm{A,t} = \pi_\mathrm{t} Y_\mathrm{t}
\end{equation}

\begin{equation} \label{eq:labor_comp}
	w_\mathrm{Y,t} L_\mathrm{Y,t} = (1-\pi_\mathrm{t}) Y_\mathrm{t}
\end{equation}

\noindent where $w_\mathrm{A}$ and $w_\mathrm{Y}$ are the wages paid to workers in the knowledge and goods producing sectors respectively (in 2005\$/year-worker), and $L_\mathrm{A}$ and $L_\mathrm{Y}$ represent the number of workers employed in each of the sectors (in workers/year). The dimensionless variable $\pi$ is the fraction of the economy's total output ($Y$) that is devoted to compensating inventors in the knowledge sector. The total labor supply ($L$) is constrained as the sum of the labor supply in the two sectors.

\begin{equation} \label{eq:labor_supply}
	L_\mathrm{t} = L_\mathrm{A,t} + L_\mathrm{Y,t}
\end{equation}

The two labor supplies can be expressed as indexed quantities when convenient.

\begin{equation}
	l_\mathrm{Y,t} \equiv \frac{L_\mathrm{Y,t}}{L_\mathrm{Y,0}}
\end{equation}

\begin{equation}
	l_\mathrm{A,t} \equiv \frac{L_\mathrm{A,t}}{L_\mathrm{A,0}}
\end{equation}

In equilibrium, it is expected that wages paid to workers in each sector will be equal.

\begin{equation} \label{eq:wage_equality}
	w_\mathrm{A,t} = w_\mathrm{Y,t} = w_\mathrm{t}
\end{equation}

The wage equality (Equation~\ref{eq:wage_equality}) allows $\pi$ to be directly proportional to the fraction of the total employed population working in the knowledge sector in static equilibrium.

\begin{equation} \label{eq:pi}
	\pi_\mathrm{t} = \frac{L_\mathrm{A,t}}{L_\mathrm{t}}
\end{equation}

Two differential equations describe the growth of knowledge and population over time. The rate of change in the indexed stock of knowledge ($a$) is given by

\begin{equation} \label{eq:da_dt}
	\frac{da}{dt} = \delta l_\mathrm{A,t}^\lambda a_\mathrm{t}^\phi
\end{equation}

\noindent where $\delta$, $\lambda$, and $\phi$ are constants, and $l_\mathrm{A}$ is the indexed labor in the knowledge sector of the economy.

The rate of change of indexed population ($n$) is a function of birth and mortality rates ($b$ and $d$ in births/year-capita and deaths/year-capita, respectively):

\begin{equation} \label{eq:dn_dt}
	\frac{dn}{dt} = b_\mathrm{t} n_\mathrm{t} - d_\mathrm{t} n_\mathrm{t},
\end{equation}

\noindent **** migration? **** where $b$ is the birth rate from Equation~\ref{eq:utility_function}, $n_\mathrm{t}$ is the indexed population, and $d_\mathrm{t}$ is the mortality rate in deaths/year-capita. The form of the mortality equation is taken from \citet{Jones:2001wn} as

\begin{equation} \label{eq:mortality_rate}
	d_\mathrm{t} = \frac{1}{\omega_\mathrm{1} z_\mathrm{t}^{\omega_\mathrm{2}} + \omega_\mathrm{3} z_\mathrm{t}} + \bar d,
\end{equation}

\noindent where $d_\mathrm{t}$ is the mortality rate and $\bar d$ is the long-run mortality rate, both in units of deaths/year-capita. $\omega_\mathrm{1}$, $\omega_\mathrm{2}$, and $\omega_\mathrm{3}$ are fitted parameters. The ratio of consumption to the subsistence level is given by $z$:

\begin{equation} \label{eq:z}
	z_\mathrm{t} = \frac{c_\mathrm{t}}{\bar c} - 1
\end{equation}

\noindent which is an indicator of the mortality rate through Equation~\ref{eq:mortality_rate}.

%++++++++++++++++++++++++++++++
\subsection{Tuning: basic model}
\label{sec:Tuning_basic_model}
%++++++++++++++++++++++++++++++

To tune the basic model we needed to determine values of each of each of the fitting parameters, or the parameters which do not vary with time in the case of the Jones model **** list the fitting parameters here ****. We sought to do this by comparing the various outputs predicted by our model to historical data and then using the least-squares optimization method. The following subsections discuss data collection and the optimization process.

\subsubsection{Data collection}
We collected historical time series data for the United States on economic output ($Y$), total labor force ($L$), knowledge sector labor force (L$_\mathrm{A}$), population ($N$), births ($b$), deaths ($d$), and wages ($w$). In all cases historical data for the United States was obtained for the years 1980-2011 unless otherwise noted. 

We measured economic output by using gross domestic product (GDP) and obtained GDP data, in real 2005 U.S. dollars per year, from the World Bank's Global Economic Prospects (GEP) database. 

We obtained total labor force data from the United States Bureau of Labor Statistics (BLS) in units of total number of workers/employed persons.

The size of the knowledge sector labor force was measured in total number of researchers, where researchers are defined as people working in research and development (R\&D) positions in STEM (science, technology, engineering, mathematics) fields. It is worth noting that this is our own definition and it differs from Jones, who does not give a particular definition on types of jobs in the knowledge sector of the economy. Data on the knowledge sector labor force was obtained from the Organisation of Economic Co-operating and Development (OECD) time series on research and development R\&D personel. Data was available from 1981-2007, with only odd years being available between 1985 and 1999, so we extrapolated forward from 2007 to 2011 and backward from 1981 to 1980.

Population was defined as the total number of people living in the United States each year. We obtained population data from the World Bank's GEP database.

We defined births as the number of births within the United States each year. We acquired data on the number of births from the Center for Diesease Control (CDC) time series on births in the United states. To obtain $b$ in the units of births/year-person, we divided the CDC birth data by the total population for each year.

We defined deaths as the number of deaths within the United States each year. We acquired data on the number of deaths from the Center for Diesease Control (CDC) time series on deaths in the United states. To obtain $d$ in the units of deaths/year-person, we divided the CDC mortality data by the total population for each year.

We defined wages as the yearly salarly that is made by someone working in either the knowledge or manufacturing sector (since the wages will be equivalent in dynamic equilibrium). Wage data was taken from the United States Census Bureau's annual population survey, which had data on average per capita income in the United States. The per capita income data was multiplied by the total population and then divided by the labor force to obtain wage data in 2005 dollars per year-worker.

Based on the historical data outlined above, we were also able to calculate ``historical'' time series for other variables. Using Equation \ref{eq:pop_work}, we calculated historical values for $\tau_\mathrm{t}$. Using the historical $\tau_\mathrm{t}$ values and Equation \ref{eq:consumption_constraint} we were able to calculate historical values of consumption. 

\subsubsection{``Ground truth'' information}
Based on the historical data we collected, we wanted to determine as much ground truth information as possible for the variables that were not meant to change over time. We used 1980 values of $\tau$ and $b$ to calculate a ground truth value for $\nu$. We also used 1980 values for $L_\mathrm{A}$ and $L$ to calculate a ground truth value for $\pi$.

To determine values of the three mortality function fitting parameters ($\omega_\mathrm{1}$, $\omega_\mathrm{2}$, and $\omega_\mathrm{3}$) we used a least-squares minimization. Historical values of z were calculated from Equation \ref{eq:z} by using the calculated historical values of consumption. These historical z values were used to calculated a predicted $d$ value that was then compared to the historical $d$ values for the minimzation. 

To determine $\eta$ and $\gamma$

\subsubsection{Optimization}




%++++++++++++++++++++++++++++++
\subsection{Results: basic model}
\label{sec:Results_basic_model}
%++++++++++++++++++++++++++++++

**** MKH to develop graphs and other results based on Caleb's existing work ****


%%%%%%%%%%%%%%%%%%%%%%%%%%%%%%%%
\section{Enhanced model (with energy and capital stock)}
\label{sec:Enhanced_model}
%%%%%%%%%%%%%%%%%%%%%%%%%%%%%%%%

%++++++++++++++++++++++++++++++
\subsection{Enhanced model development}
\label{sec:Enhanced_model_development}
%++++++++++++++++++++++++++++++

We now enhance the basic model with both energy flows and capital stock.

\begin{figure} \label{fig:ModelWithEnergy}
  \begin{center}
    \includegraphics[width=\textwidth]{figure_other/ModelWithEnergy.pdf}
    \caption{Dynamic model with energy.}
  \end{center}
\end{figure}

%++++++++++++++++++++++++++++++
\subsection{Tuning: enhanced model}
\label{sec:Tuning_enhanced_model}
%++++++++++++++++++++++++++++++

**** Caleb: discuss the process you used to tune the enhanced model. ****

%++++++++++++++++++++++++++++++
\subsection{Results: enhanced model}
\label{sec:Results_enhanced_model}
%++++++++++++++++++++++++++++++

**** MKH to develop graphs and other results based on Caleb's existing work ****


%%%%%%%%%%%%%%%%%%%%%%%%%%%%%%%
\section{Implications}
\label{sec:Implications}
%%%%%%%%%%%%%%%%%%%%%%%%%%%%%%%

%%%%%%%%%%%%%%%%%%%%%%%%%%%%%%%
\section{Conclusion}
\label{sec:Conclustion}
%%%%%%%%%%%%%%%%%%%%%%%%%%%%%%%


%%%%%%%%%%%%%%%%%%%%%%%%%%%%%%%
%% Bibliography (via BibTeX)
%%%%%%%%%%%%%%%%%%%%%%%%%%%%%%%
\bibliographystyle{DeGruyter}
\bibliography{ReeseHeunJournalPaper}

\end{document}