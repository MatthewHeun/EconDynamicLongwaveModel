\documentclass[letterpaper,12pt]{article}
\usepackage{dgjournal}          
\usepackage{mathptmx}
\usepackage{graphics}
\usepackage[authoryear,comma,longnamesfirst,sectionbib]{natbib} 

\begin{document}

%% Do NOT include any fronmatter information; including the title, author names,
%% institutes, acknowledgments and title footnotes (author information, funding
%% sources, etc.). Start the document with the first section or paragraph of
%% the article.

\section{Mathematical model from Jones (2001)}

The mathematical model being used for this project was taken from Jones (2001). The production function that Jones uses is defined in equation \ref{eq:Jones_production_function}.
\begin{equation} \label{eq:Jones_production_function}
y = a_\mathrm{t} ^\sigma l_\mathrm{Yt} ^\beta t_\mathrm{t} ^{1-\beta} \epsilon
\end{equation}

Definitions of the production function variables, indexed against some base year.
\begin{equation} \label{eq:index_y}
	y \equiv Y/Y_\mathrm{0}
\end{equation}
\begin{equation} \label{eq:index_a}
	a \equiv A/A_\mathrm{0}
\end{equation}
\begin{equation} \label{eq:index_t}
	t \equiv T/T_\mathrm{0}
\end{equation}

The number of people 100 percent commited to labor in the economy.
\begin{equation}\label{eq:pop_work}
	L = \tau_\mathrm{t} N
\end{equation}

The number of people 100 percent commited to producing children.
\begin{equation} \label{eq:pop_birth}
	N-L = (1-\tau_\mathrm{t})N
\end{equation}

Time constraints in the economy are represented by the two following equations.
\begin{equation} \label{eq:consumption}
	c_\mathrm{t} = w_\mathrm{t} \tau_\mathrm{t}
\end{equation}

\begin{equation} \label{eq:birth}
	b_\mathrm{t} = \alpha (1-\tau_\mathrm{t})
\end{equation}

First-order condition.
\begin{equation} \label{eq:first_order_condition}
	\frac{u_\mathrm{\tilde b}}{u_\mathrm{\tilde c}} = \frac{w_\mathrm{t}}{\alpha}
\end{equation}

Equations representing compensation for innovation in the knowledge and labor sector respectively.
\begin{equation} \label{eq:knowledge_comp}
	w_\mathrm{At} L_\mathrm{At} = \pi_\mathrm{t} Y_\mathrm{t}
\end{equation}

\begin{equation} \label{labor_comp}
	w_\mathrm{Yt} L_\mathrm{Yt} = (1-\pi_\mathrm{t})Y_\mathrm{t}
\end{equation}

Differential equation representing the change in the stock of knowledge over time.
\begin{equation} \label{eq:da_dt}
	\frac{da}{dt} = \delta l_\mathrm{At}^\lambda a_\mathrm{t}^\phi
\end{equation}

The utility function for choosing between consumption and having children.
\begin{equation} \label{eq:utility}
	u(c_\mathrm{t}, b_\mathrm{t}) = (1-\mu) \frac{\tilde c_\mathrm{t}^{1-\gamma}}{1-\gamma} + \mu \frac{\tilde b_\mathrm{t}^{1-\eta}}{1-\eta}
\end{equation}

\begin{equation} \label{eq:c_tilde}
	\tilde c_\mathrm{t} \equiv c_\mathrm{t} - \bar c
\end{equation}

\begin{equation} \label{eq:b_tilde}
	\tilde b_\mathrm{t} \equiv b_\mathrm{t} - \bar b
\end{equation}

Differential equation representing the change in the population over time.
\begin{equation} \label{dn_dt}
	\frac{dn}{dt} = b_\mathrm{t} n_\mathrm{t} - d_\mathrm{t} n_\mathrm{t}
\end{equation}

Death rate equation.
\begin{equation} \label{deathrate}
	d_\mathrm{t} = \frac{1}{\omega_\mathrm{1} z_\mathrm{t}^{\omega_\mathrm{2}} + \omega_\mathrm{3} z_\mathrm{t}} + \bar d
\end{equation}

\subsection{A second order heading}

Some text under the subheading. Paragraphs that follow heads are not
indented.

Math should also be set in Times. Use the mathptmx package if you do not have
any of the commercially available fonts that are compatible with Times.
\begin{equation}
    y^{(n)} = \sum_{i=0}^{n-1} a_i(x) y^{(i)} + r(x) 
\end{equation}

All environments provided by the standard LaTeX document classes are
unchanged. Vertical spaces within lists have been altered to comply with De Gruyter
requirements.
\begin{enumerate}
\item This is the first item within the list. Some more text here in order to
  display the alignment.
\item Another item in the list.
\item Yet another item in the list.
\end{enumerate}

Here is an example of a Figure. It's the same as in standard LaTeX.

\begin{figure}[!h]
%% Use the graphics package to insert figures
%% \includegraphics{figure.eps}
% Use \centering to center the table
\centering
%% A small box in place of a figure
\framebox{%
  \begin{minipage}{10pc}
    \begin{center}
      \vspace{1cm}\par
      A figure\par
      \vspace{1cm}
    \end{center}
  \end{minipage}}
\caption{Insert your caption here. If you wish to label your figure for
  cross-referencing, use a label either within the caption or after it.}
\label{fig1}
\end{figure}

An example of a table follows. This is also the same as in standard LaTeX.

\begin{table}[!h]
% Use \centering to center the table
\centering
\caption{Insert your table caption here. If you wish to label the table for
  cross-referencing, use a label either within the caption or after it.}
\begin{tabular}{llll}
\hline
Symbol        & LaTeX Command      & Symbol      & LaTeX Command \\
\hline
$\alpha$      & \verb+\alpha+      & $\zeta$     & \verb+\zeta+ \\
$\beta$       & \verb+\beta+       & $\eta$      & \verb+\eta+ \\
$\gamma$      & \verb+\gamma+      & $\theta$    & \verb+\theta+ \\
$\delta$      & \verb+\delta+      & $\vartheta$ & \verb+\vartheta+ \\
$\epsilon$    & \verb+\epsilon+    & $\iota$     & \verb+\iota+ \\
$\varepsilon$ & \verb+\varepsilon+ & $\kappa$    & \verb+\kappa+ \\
\hline
\end{tabular}
\end{table}

Use the \verb+thebibliography+ environment for the references.  BibTeX users may
use the provided BibTeX style file DeGruyter.bst.

%% BibTeX support
\bibliographystyle{DeGruyter}
\bibliography{sample}

\end{document}